\documentclass{article}
\usepackage{algorithm, algpseudocode}
\usepackage{geometry}
\usepackage{amsmath, amssymb, amsthm, enumerate, hyperref}
\usepackage{color}
\usepackage{setspace}
\usepackage{fancyhdr,lastpage}
\usepackage{url}
\usepackage{tabularx}
\pagestyle{fancy}
\lhead{\footnotesize Week 4 Proof Writeup (Sept 8)}
\chead{}
\rhead{\footnotesize CSC (791/495)-011 -- Fall 2017}
\lfoot{}
\cfoot{\small \thepage/\pageref*{LastPage}}
\rfoot{}

\title{Homework 2}
%\author{Linfeng Zhou}


\newcommand{\defproblem}[4]{%
  \hfill\\\smallskip\noindent%
  \begin{tabularx}{\textwidth}{|l X|}%
    \hline%
    \multicolumn{2}{|l|}{\pname{#1}}\\%
    \textbf{Input:}&#2\\%
    \textbf{Parameter:}&#3\\%
    \textbf{Question:}&#4\smallskip\\\hline%
  \end{tabularx}%
  \smallskip%
}%

\newcommand{\pname}[1]{\textnormal{\textsc{#1}}}

\newtheorem*{theorem}{Theorem}
\newtheorem{definition}{Definition}
\newtheorem*{lemma}{Lemma}

\title{Homework 3}
%\author{Linfeng Zhou}


\begin{document}
\maketitle


\subsection*{Definitions \& Notation}
\begin{definition}
    A \emph{proper vertex coloring} of a graph $G$ \emph{using $k$ colors}
    is a function $c: V(G) \rightarrow \{1,\ldots, k\}$ so that if $uv \in E(G)$,
    $c(u) \neq c(v)$.
\end{definition}


\defproblem{$k$-Coloring-VC}
%
{A graph $G$, a vertex cover $X$ of $G$ with $|X| = \ell$, and a non-negative integer $k$}
%
{$\ell$}
%
{Is there a proper vertex coloring of $G$ using at most~$k$ colors?}
%


\subsection*{Proofs Required}

\begin{theorem} $k$-\pname{Coloring-VC} is in FPT (there is an $f(\ell)n^{O(1)}$ algorithm). 
\end{theorem}

\begin{proof}
We prove that \(k\)-\pname{Coloring-VC} is in FPT through the following steps. Let \(n\) be the number of vertices in the graph \(G\). 
\begin{enumerate}
\item First of all, we can remove those independent vertices in the graph \(G\) because we can fill any color since these vertices do not have any neighbors. 
\item Decompose the vertex cover \(X\) and the rest of graph \(G\), which is denoted by \(G-X\). This step takes time linear in \(n\), where \(n\) is the number of vertices of the graph \(G\). 
\item Then we check each vertex \(v\) in the subgraph \(G-X\) that whether \(v\) has less than \(k-1\) neighbors that are vertices in the vertex cover \(X\). Here we denote the set of neighbors, which are vertices in \(X\), of the vertex \(v\) as \(N\). The intuition of this step is the observation as follows.
    \begin{enumerate}
    \item If the vertex \(v\) has less than \(k-1\) neighbors that are vertices in the vertex cover \(X\), then the set \(N \cup \{v\}\) must be \(k\)-colorable. Then we fill colors for each vertex in \(N \cup \{v\}\).
    \item If the vertex \(v\) has at least \(k\) neighbors that are vertices in the vertex cover \(X\), we need to check that whether the set \(N \cup \{v\}\) is \(k\)-colorable or not. If the set \(N \cup \{v\}\) is not \(k\)-colorable, output \(\bot\). If \(N \cup \{v\}\) is \(k\)-colorable, then we fill the color for each vertex in \(N \cup \{v\}\). There are \((\ell+1)^k\) possibilities using brute-force method. 
    \end{enumerate}
    Since each vertex in \(G-X\) is connected to one or more vertices in \(X\) due to the definition of vertex cover and for every two vertices in \(G-X\) they do not have connection to each over, the method above considers each vertex in the graph \(G\) and it is safe.
\end{enumerate}
We focus on analyzing the (worst-case) running time of step 3. Since the size of \(G-X\) is \(n-\ell\) and the size of \(X\) is \(\ell\). Therefore, the worst-case running of the above algorithm is \((n-\ell)\cdot \ell \cdot (\ell+1)^k + O(n)\), since \(k \leq \ell\), we have \((n-\ell) \cdot \ell \cdot \ell^k \leq (n-\ell) \cdot \ell \cdot (\ell+1)^\ell = (\ell+1)^\ell \cdot \ell \cdot n - (\ell+1)^\ell \cdot \ell^2 = O((\ell+1)^\ell \cdot \ell \cdot n)\), which is in FPT.




\end{proof}

%\vfill
%
%\subsubsection*{Submission notes}
%\begin{itemize}
%\item Due at 9:00 am on Friday, September 8.
%\item Use the \href{https://github.com/bdsullivan/ParameterizedAlgorithms-Fall2017/tree/master/templates/homework}{latex homework template}.
%\item Please put your name in a comment at the top of your tex file (but {\it do not
%have it display in the PDF}). This will help prevent any bias in grading.
%\item Keep all files in \texttt{Week-3} folder of git repository
%\item Name source file \texttt{homework.tex}
%\item Upload a compiled version named \texttt{compiled\_homework.pdf}
%\end{itemize}

\end{document}
